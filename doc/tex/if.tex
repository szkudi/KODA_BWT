%% !TeX root = report.tex

Algorytm IF (ang. Inversion Frequencies)\cite{deorowicz} został zaproponowany przez Arnavut i Magliveras w roku 1997. Jest on stosowany jako zamiennik MTF (\ref{ch:mtf}) w algorytmie kompresji BWT. Algorytm ten jako wyjście podaje sekwencje liczb z przedziału $[0; n -1]$, gdzie $n$ jest ilością elementów w ciągu wejściowym. Dla każdego elementu $a_i$ wykorzystywanego alfabetu algorytm skanuje sekwencje wejściową. W momencie pierwszego napotkania aktualnie analizowanego znaku $a_i$ algorytm wypisuje jego pozycję. Dla każdego kolejnego wystąpienia na wyjściu zapisywana jest liczba będąca ilością znaków większych niż analizowany element  $a_i$, która pojawiła się od jego ostatniego wystąpienia. Powstała w ten sposób sekwencja nie jest niestety wystarczająca do odkodowania danych wejściowych. Drugim niezbędnym elementem jest lista zawierająca ilość wystąpień każdego ze znaków w analizowanym ciągu.