%% !TeX root = report.tex

Algorytm kodowania dystansowego (ang. Distance Coding - DC)\cite{deorowicz} został zaprezentowany w 2000 roku przez Edgara Bindera. Nie ma żadnej oficjalnej publikacji na jego temat. Został on zaproponowany na jednej z grup dyskusyjnych o kompresji danych. 

Dla każdego symbolu wejściowego $x_i^{bwt}$ algorytm znajduje odległość do jego następnego wystąpienia i ją zapisuje na wyjście. Jeżeli jest to ostanie wystąpienie elementu w ciągu wejściowym to na wyjście wypisywane jest $0$. Do poprawnego odczytania o ponownego zdekodowania zbioru wyjściowego niezbędna jest znajomość pozycji początkowych wszystkich elementów alfabetu występujących w analizowanym ciągu.