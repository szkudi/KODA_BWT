%% !TeX root = report.tex

Move To Front (MTF) \cite{deorowicz} jest to prosta transformacja strumieniowa, której celem jest zmniejszenie entropii w kodowanym ciągu danych. Algorytm ten jest zalecany przez twórców BWT jako algorytm drugiego stopnia, najlepiej współpracujący z BWT. Algorytm operuje na danych wejściowych i liście $L$ zawierającej wszystkie elementy alfabetu. Początkowo lista $L$ jest posortowana w pewien ustalony sposób. Algorytm odczytuje z wejścia kolejne symbole ciągu wejściowego i dla każdego z nich na wyjście wyprowadzana jest pozycja odpowiadająca temu elementowi w liście $L$. Następnie lista jest modyfikowana tak, że element, który wystąpił jako ostatni wędruje na pierwszą pozycję. Taka modyfikacja powstanie ciągów powtarzających się elementów na wyjściu. Np. wszystkie ciągi jednakowych elementów zostaną zastąpione przez ciągi zer i jednej dodatkowej liczby stanowiącej pozycję początkową znaku. Do zdekodowania otrzymanego ciagu wyjściowego potrzebna jest jedynie wiedza na temat początkowego posortowania listy $L$ zawierającej alfabet.