%% !TeX root = report.tex
%% 
Zbiór {\em artificl} zawiera cztery sztucznie wygenerowane pliki
zawierające odpowiednio: jedną literę a {\em (a.txt)}, powtarzające się
litery a {\em (aaa.txt)}, powtarzający się alfabet {\em (alphabet.txt)} oraz
litery w kolejności pseudolosowej {\em (random.txt)}. Może być on pomocny w
sprawdzeniu jak algorytmy zachowują się w najgorszym możliwym
scenariuszu (losowe znaki, bardzo mały plik).

Zbiór {\em calgary} zawiera 18 plików różnego typu stanowiących {\em The Calgary
Corpus}, standard dla badania wydajności bezstratnej kompresji w latach
90. ubiegłego wieku.

Zbiór {\em cantrbry} zawiera 11 plików stanowiących {\em The Canterbury Corpus}.
Jest to zbiór utworzony w 1997 roku jako udoskonalenie poprzedniego
zbioru. Pliki zostały tak dobrane, że ich rezultat przy wykorzystaniu
ówczesnych algorytmów kompresji był reprezentatywny i stąd nadzieja, że
będzie on taki także dla nowych metod kompresji.

Zbiór {\em large} zawiera 3 duże pliki (od 2 do 4,4 MB każdy) --- kompletny
genom bakterii E. Coli, jedna z wersji Biblii oraz opis krajów świata
--- {\em The CIA world fact book}.