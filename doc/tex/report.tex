\documentclass[a4paper,12pt]{article}


\usepackage[MeX]{polski}
\usepackage[utf8]{inputenc}	% kodowanie znaków
\usepackage{indentfirst}

\usepackage{upquote}  	% zamiana cudzysłowów klasycznych (,, ") na górne (" ") w kodach źródłowych
\usepackage{graphicx}	% wstawianie obrazków
\usepackage{amsmath}
\usepackage{listings}


\def\PICSDIR{PICS}

% Ustawienie marginesów
\oddsidemargin=0.5cm
\evensidemargin=-0.5cm
\topmargin=0cm
\textwidth=16cm
\textheight=23cm

% Ładniejsze tabelki
\usepackage{booktabs}


\frenchspacing

\clubpenalty=10000		% to kara za sierotki
\widowpenalty=10000		% nie pozostawia wdów
\brokenpenalty=10000 	% nie dzieli wyrazów pomiędzy stronami


\sloppy

% Pakiety z ładnymi czcionkami
\usepackage[T1]{fontenc}
\usepackage{lmodern}


% Ładniejsze tabelki
\usepackage{booktabs}

% Ustawienia wyglądu listingów
\lstset{
	language=C++,                               % choose the default language of the code
	basicstyle=\footnotesize\ttfamily,       	% the size of the fonts that are used for the code
	numbers=left,                   		% where to put the line-numbers
%	numberstyle=\tiny,      			% the size of the fonts that are used for the line-numbers
	stepnumber=1,                   		% the step between two line-numbers. If it's 1 each line will be numbered
	numbersep=5pt,                  		% how far the line-numbers are from the code
	showspaces=false,               		% show spaces adding particular underscores
	showstringspaces=false,         	% underline spaces within strings
	showtabs=false,                 		% show tabs within strings adding particular underscores
	tabsize=4,	                		% sets default tabsize
	captionpos=b,                   		% sets the caption-position
	breaklines=true,                		% sets automatic line breaking
	breakatwhitespace=true,        	% sets if automatic breaks should only happen at whitespace
	extendedchars=true,
	keywordstyle=\bfseries,
	identifierstyle=,
	commentstyle=\slshape,
	stringstyle=\slshape,
	xleftmargin=30pt,
	frame=tb,
	framexleftmargin=30pt,
}

\begin{document}

\title{{\small Kompresja danych}\\Implementacja BWT oraz algorytmów dodatkowych}
\author{Jakub Machoń, Łukasz Banaśkiewicz, Kacper Szkudlarek}

\maketitle

\begin{abstract}
Transformata Burrowsa-Wheelera to algorytm użyteczny przy bezstratnej kompresji danych. Dane po przetworzeniu tą transformacją dają się znacznie lepiej skompresować za pomocą klasycznych algorytmów kompresji. W ramach projektu powstała implementacja transformaty oraz kilku algorytmów drugiego kroku wykorzystywanych razem z BWT.
\end{abstract}


\section{Wstęp}


\section{Implementacja}

\subsection{Transformata Burrowsa-Wheelera}

\subsection{RLE0 i RLE2}

\subsection{Increment Frequency Count}

\subsection{Distance Codding}

\subsection{Move To Front}

\subsection{Invesrion Frequencies}

\section{Testy}

\subsection{Aplikacja testująca}

\subsection{Testy wydajności}

\subsubsection{Dane testowe}

\subsubsection{Wyniki}

\end{document}
